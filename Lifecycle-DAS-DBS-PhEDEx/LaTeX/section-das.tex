\subsection{Data Aggregation Service}
The Data Aggregation Service (DAS) \cite{DAS} is a critical component of the CMS Data
Management System. It provides the ability to query CMS data-services via a uniform
query language without worrying about security policies and differences in
underlying data representations. When a physicist poses a question like {\it ``Find me files from RelVal dataset at certain site"}, DAS converts the question into a
series of requests to the underlying data-services. DAS takes care of aligning naming conventions, data conversion into common
JSON format and aggregating data on a requested entity key, e.g. file names.
As a result, the user gets back an aggregated record across participating
data-providers.  This workflow fits really well into LifeCycle design which
utilize DAS as cross-check validation tool for its components.

% This is redundant with section 6, on the integration tests
%A LifeCycle workflow was configured to use PhEDEx, DBS and DAS. Data were
%injected into PhEDEx and DBS, then retrieved in DAS and presented back to the
%workflow. Since the original data were generated within the workflow, it was
%able to compare the initial data with the output from DAS. This represents a
%unit test for data injection and retrival through different systems. We
%compared the following set of data:
%
%\begin{itemize}
%\item dataset injection into DBS system
%\item block and files injection into DBS and PhEDEx systems
%\item site content represented in PhEDEx system
%\end{itemize}