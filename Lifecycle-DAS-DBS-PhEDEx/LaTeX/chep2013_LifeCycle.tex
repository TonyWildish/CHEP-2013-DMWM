\documentclass[a4paper]{jpconf}
\usepackage{graphicx}
\usepackage{color}
\usepackage{array}
\usepackage{enumerate}

% for DAS section
\usepackage{amsmath}
\usepackage{amstext}
\usepackage{amssymb}
\usepackage{graphicx}

\begin{document}
\title{Integration and validation testing for PhEDEx, DBS and DAS with the PhEDEx LifeCycle agent}

\author{C Boeser$^1$, T Chwalek$^1$, M Giffels$^2$, V Kuznetsov$^3$
        and T Wildish$^4$}

\address{$^1$ Institut f\"ur Experimentelle Kernphysik, Karlsruhe, Germany}
\address{$^2$ PH-CMG-CO, CERN, CH-1211 Gen\`eve 23, Switzerland}
\address{$^3$ Cornell University, Ithaca, NY, USA }
\address{$^4$ Princeton University, Princeton, NJ, USA }

\ead{a.wildish@princeton.edu}

\begin{abstract}
The ever-increasing amount of data handled by the CMS dataflow and workflow management tools poses new challenges for cross-validation among different systems within CMS experiment at LHC. To approach this problem we developed an integration test suite based on the LifeCycle agent, a tool originally conceived for stress-testing new releases of PhEDEx, the CMS data-placement tool. The LifeCycle agent provides a framework for customising the test workflow in arbitrary ways, and can scale to levels of activity well beyond those seen in normal running. This means we can run realistic performance tests at scales not likely to be seen by the experiment for some years, or with custom topologies to examine particular situations that may cause concern some time in the future.

The LifeCycle agent has recently been enhanced to become a general purpose integration and validation testing tool for major CMS services (PhEDEx, DBS, DAS). It allows cross-system integration tests of all three components to be performed in controlled environments, without interfering with production services.

In this paper we discuss the design and implementation of the LifeCycle agent. We describe how it is used for small-scale debugging and validation tests, and how we extend that to large-scale tests of whole groups of sub-systems. We show how the LifeCycle agent can emulate the action of operators, physicists, or software agents external to the system under test, and how it can be scaled to large and complex systems.
\end{abstract}

\section{Introduction}
PhEDEx \cite{PhEDEx}, DBS \cite{DBS} and DAS \cite{DAS} are the core components of the CMS Data 
Management system \cite{CMSDMS}. PhEDEx is responsible for moving data around the CMS computing 
sites, DBS is responsible for maintaining knowledge of the physics content of the data, and DAS is 
a portal that provides physicists with integrated access to PhEDEx, DBS, and other services.

The development model for these services is highly decoupled. Each is developed by a separate team 
of people, and each provides a web-based data service for access to information. Ensuring 
stability and consistency of these interfaces has, until recently, been poorly managed. We have 
relied on stability of the underlying services and coordination between developers to ensure that 
successive releases of the software are coherent and correct.

In this paper we describe part of the solution to this problem. The {\it LifeCycle agent}, written 
originally for scale-testing of PhEDEx, has been extended to provide cross-component integrated 
testing in a flexible and powerful manner. Components can be tested individually or together in 
different ways for verification, debugging, or scale and performance testing. The LifeCycle agent 
itself scales well, and is easily adapted to new requirements (e.g. testing of new APIs from the 
services). It can inject data into the component systems, retrieve it, manipulate it, and compare 
it with expectations to detect errors. It can also test expected error conditions for correct 
behaviour (e.g. access with an unauthorised certificate).


\section{The LifeCycle agent}
\subsection{Architecture}
The LifeCycle agent emulates the actions of users or external processes in 
a system. The key abstraction is the {\it workflow}, which consists of a sequence of {\it events} 
that depend on each other and are performed at distinct intervals. A sample workflow for PhEDEx could consist of the following events:

\begin{itemize}
  \item Generate data - a dataset is generated with a set of files.
  \item The files are `injected' (registered) into PhEDEx at a given site.
  \item The datasets are subscribed for transfer to one or more other sites.
  \item After a time, the data may be deleted from one or more of the sites.
\end{itemize}

By varying the details of how the generated data is organised, which sites it is sent to, and 
which sites (if any) it is finally deleted from, we can emulate different real-life situations.

Many workflows can run in parallel. We can emulate the production of dozens of 
datasets, each destined for different sites, each with different characteristics, to give us a 
realistic emulation of the total workload of the production transfer system.

\subsection{Event handlers}
Each workflow has an array of {\it events}. This is an array of strings which the agent maps to their handlers. Handlers can be Perl modules loaded at runtime, or external scripts written in any language.

To use an external script as a handler, it must accept two filename arguments, one for input and one for output. The LifeCycle agent packages the workflow as a JSON object, writes it to the input file, then calls the script. The script reads the file and reconstitutes the object in it's own representation (e.g. a python dictionary), then accesses the elements it needs to do its work. When the script finishes it re-packages the workflow and writes it to the output file, which the LifeCycle agent will read and use to update its internal representation of the workflow.

Workflows are processed in parallel. As soon as one event-handler for a workflow finishes, the next event for that workflow is queued, for execution after a configurable delay. External script handlers are run asynchronously, handlers written as Perl modules can be either synchronous or asynchronous. A workflow completes when its last event has been processed. After that, depending on the configuration for that workflow, it can be restarted again at a later time.

Mechanisms are provided for event handlers to signal errors to abort the workflow or the entire LifeCycle agent, or to provide statistics that are accumulated by the agent and reported periodically.

Workflows need not be fixed. The event handler has access to the entire workflow object and can manipulate the event chain in many ways. It can re-insert the current event, to block the workflow, polling for an external condition to be satisfied. It can append events to the event chain, to steer the workflow according to the conditions it has encountered so far. Handlers can even spawn new workflows, by cloning and modifying their own workflow.

\section{PhEDEx}
\subsection{Scale tests of PhEDEx}


\subsection{Testing new PhEDEx releases}

The LifeCycle agent is used to test PhEDEx releases. For this purpose a
private PhEDEx instance is set up in a testbed with a few nodes, and transfers
between these nodes are driven using the LifeCycle agent to orchestrate the
flow. In this procedure bug fixes and new features of new PhEDEx versions can
be tested before releasing the version. 

For special cases, such as bugs that occur under unusual consitions, the
LifeCycle agent can be configured to reproduce these conditions, repeatedly
and systematically.


\subsection{Integration test of the webbased PhEDEx DataService}


One of the harder things to test is the web-site access by different users
with different access-rights. Rather than requiring the support of several
people to systematically test the new release of the website, with all the
coordination and overhead that implies, we can use the LifeCycle agant again.

We extend the private testbed to include a website, with self-signed
certificates allocated to specific roles. The LifeCycle agent uses the website
to perform a series of actions (create request, approve it, change its
priority), repeating this workflow for each certificate, and for the different
source and destination nodes. It checks that actions succeed where they should
and fail where appropriate. In this way, several combinations of actions
(inject, subscribe,...), destination nodes (T0, T1,...), and roles
(site-admin, data-manager,...) can be tested in an automated way by
configuring and running one single script.




\section{DBS}
%!TEX root = chep2013_CMS-Data-Management.tex
The Data Bookkeeping Service (DBS) \cite{DBS} provides a catalogue of event metadata for Monte Carlo and recorded data of the CMS experiment. DBS contains record of what data exist, their process-oriented provenance information, including parentage relationships of files and datasets, their configurations of processing steps, as well as associations with run numbers and luminosity sections to find any particular subset of events within a dataset, on a large scale of about $200,000$ datasets and more than $40$ million files, which adds up to around $700$ GB of metadata.

The current DBS, DBS~2 \cite{DBS2} was designed in 2006-2007, before the LHC started its operation. CMS did not have a standardised service architecture for the implementation, deployment and operation of that kind of web services. Thus DBS~2 was implemented using Java servlets in an Apache Tomcat container and XML RPC has been the first choice for the client-server communication. As  persistent storage system of the metadata, an Oracle database backend provided for CMS \cite{CMSDBs} is utilised. DBS~2 additionally supports a MySQL database backend, however it is currently not used in the production environment.

The DBS is a federated system with multiple instances for different scopes. Driven by the principle in CMS to separate official and user-created data, the Global-DBS contains metadata related to all official CMS data, real or simulated. Metadata related to user-created datasets can be stored in two physics analysis DBS instances. Besides those instances there are also a CAF instance, that records data from the prompt reconstruction stream used for detector calibrations and diagnostics, and a CMS Tier0 DBS instance, that records information for data from the detector as it is processed by the Tier0 facility.

Although DBS~2 sustained the load in LHC Run $1$, it has in some cases very ``thick'' client APIs, which  led to numerous problems with API versioning and scalability issues. In addition, the CMS data processing model has evolved a lot, in a way that could not be anticipated by the DBS~2 design back in $2006$, so many requests were made to store additional data in DBS~2, which were not entirely consistent with its original purpose. A project review in $2009$ led to the decision to re-design DBS, to better match the CMS data processing model and to better integrate with the DMWM projects.

The CMS DMWM project has meanwhile developed a standardised architecture based on the Representational State Transfer API (REST) \cite{REST} for its web services based on Python, CherryPy and SQLAlchemy. Thus DBS~3 has been re-designed and re-implemented in Python utilizing the CMS DMWM standards for RESTful web services. The client-server communication is stateless and REST also imposes the discipline of thin clients, which enhances the scalability of the service. The Java Script Object Notation (JSON) is used as data-format for the client-server communication being a lightweight replacement for XML RPC. The database schema of DBS~3 also has been revised, based on the experiences with DBS~2. The schema has been denormalised, since the DBS~2 schema was fully normalised, leading to excessive table joins and lock contention. In DBS~3 some tables were removed, as well as some relationships that were better modeled outside of the DBS schema. These changes sped up searches and also improved the insertion of data by removing foreign keys and lock contention. In addition, the DBS~3 particularly benefitted from the narrower and more precise project scope compared to DBS~2. The integration with other services (PhEDEx and DAS) in the CMS DMWM project made a narrower scope possible, without impairing the query features, so for example the data location is not anymore stored in DBS, since it is naturally available from PhEDEx. DBS~3 is currently deployed in parallel to DBS~2 for validation and integration with other DMWM projects. The final switch to DBS~3 is expected by the end of $2013$.



\section{DAS}

The Data Aggregation Service (DAS) \cite{DAS} was designed to provide a uniform
access to distributed CMS data-services regardless of their security policies,
implementation details and data storage solution. DAS provides users with the ability to
query underlying data-services via the DAS Query Language (QL). It also aggregates
metadata information from various data-services into common records
suitable for end users. Here we discuss the details of DAS architecture and the
current status of the system.

DAS was built on top of a NoSQL document-oriented database (MongoDB
\cite{MongoDB}). It provides several benefits for the DAS use case: schema-less
storage, embedded query language and very fast read/write
operations.\footnote{Our benchmark showed that MongoDB can sustain 20k doc/s
for reading and 7k doc/s for writing operations, respectively.}
DAS has a modular design based on the MVC architecture \cite{MVC} and performs the
following workflow upon user query:

\begin{itemize}
\item fetch data from underlying data-services via existing APIs
\item store unprocessed data into the cache database
\item process data and perform its transformation to a common data representation
(we unify differences in naming conventions, units and data-formats, etc.)
\item aggregate data on a requested key, e.g. dataset, block, etc.
\item store results into the merge database
\item present results to end-users and provide a set of filters as well as
aggregation functions to perform basic operations for slicing and representing
the data in a form suitable for user tasks
\end{itemize}

This design allowed to develop DAS in a data-agnostic manner. For instance,
data can be fetched from underlying data-services regardless of their data format, e.g.
JSON, XML, CSV, etc. The data transformation was done via an external set of
mappings which was maintaned separately from DAS code development. It also gave
us a few benefits which were not foreseen from the design cycle. For example,
data aggregation on a common entity, e.g. data name, may show any existing discrepancy in
underlying services.

DAS uses an in-house Query Language which was based on entity relationships used by
physicists, see \cite{DAS}. It consists of the following structure:

\begin{verbatim}
<selection keys> <set of conditions> | <filters> | <aggregators>
\end{verbatim}

The selection keys were based on well known entities such as dataset, block, file,
run. The conditions were formed via key=value pairs. DAS-QL
provides a limited set of filters such as grep, unique,
as well as set of common aggregation functions such as
sum, min, max. The former support conditional operators and grouping, while
the latter can be extended via custom map-reduce functions to support more complex
use cases. Therefore to express a question {\it Find me all datasets at a
given site and show only those which have size greater than 50} someone will need to
write a DAS query in the following way:

\begin{verbatim}
dataset site=XYZ | grep dataset.name, dataset.size>50
\end{verbatim}

It turns out that such flexibility was not always clear to some users,
mostly those who were unfamiliar with the DAS-QL syntax. Therefore a further
attempt was made to build native support for keyword queries on top of DAS-QL,
see \cite{DASKWS}.

DAS operates as an intelligent cache in front of CMS data-services. It stores
results into two caches upon a provided query. The raw-cache is used to store results from
data-services {\it as is}, while the merge-cache stores aggregated
records. The lifetime of the records is based on information provided by
data providers via HTTP headers. The record maintenance is done in a lazy
fashion, i.e. upon a new user query expired records are wiped out from the cache,
while new ones come in. DAS server performs many operations in parallel, e.g.
it sends concurrent HTTP requests to underlying data-services, processes and
stores
data using multiple threads, and runs multiple monitoring and pre-fetching
daemons. The server runs on a single 8-core node with 24 GB of memory
required for efficient MongoDB operations\footnote{MongoDB relies on indexes
fitted in RAM to provide its superior speed}.

The discussed modular design, flexible QL and NoSQL storage allows DAS to
aggregate information from distributed data-services without imposing any
requirements on them. DAS is able to deal with different security models,
various APIs, data-formats and naming conventions. Right now it uses dozens of CMS
data-services. Data are aggregated into JSON records based on common entity
keys so that it is possible to see information from multiple
data-services in a single record, e.g. run information comes from DBS, Condition DB and
RunRegistry and is represented as a single JSON document listing information from
three data-services. Daily load on the DAS server is constanly growing and has about 10k
queries/day with $\sim O(10M)$ records going in and out of the DAS cache.



\section{Integration testing of PhEDEx, DBS, and DAS}
The LifeCycle agent was used for cross-integration tests of all three
components (PhEDEx, DBS 3, and DAS) within a controlled environment and thus
without interfering with production services. Fake data was created and injected into PhEDEx and DBS 3. DAS then
retrieved information about the data from both sources, and compared the
results. For an illustration of the workflow see Fig.~\ref{fig:Integ-Phedex-DBS-DAS}. By injecting specific errors in either PhEDEx or DBS 3 (changing filenames etc)
we can fake errors that we expect to detect with DAS. Special event-handlers
are used to compare the errors detected by DAS with the injected errors, and
alert us to any unexpected failures.

In the current implementation six kinds of failure can be faked and the probability to occur
can be adjusted for each single failure type individually.

\begin{itemize}
\item Single file is skipped in the PhEDEx data
\item Single file is skipped in the DBS 3 data
\item The checksum for a single file is changed for the PhEDEx data
\item The checksum for a single file is changed for the DBS 3 data
\item The size for a single file is changed for the PhEDEx data
\item The size for a single file is changed for the DBS 3 data
\end{itemize}

The payload-provider (data provider) generates fake data, which then can be
injected to a node of the LifeCycle testbed. It also
generates extensions to filenames specifying the failure type the file is
affected by (according to the probability that was specified).

Special modules read the filenames and act on all files with a certain
extension specifying the kind of failure the file is affected by. This
includes whether the PhEDEx information, the DBS 3 information,
or both are affected and the kind of failure (different checksum, different
size or skipped file). Also combinations of these failures are possible for
one single file. According to the kind of failure these modules either
remove the files, change their size or their checksum, or perform a
combination of the mentioned actions, when injecting the
information to PhEDEx and DBS 3.

After passing these error-handler modules, all files are injected into PhEDEx
and DBS 3. A DAS-query is performed retrieving the information on the
injected data from PhEDEx and from DBS 3.
These two sets of information on the injected data are compared and the differences found
are checked against the true information generated by the workflow.
In this way we can check that DAS properly reports all occuring mismatches in
the information on a certain datset coming both from PhEDEx and/or DBS 3.

\begin{figure}[h]
 \centering
   \includegraphics[width=0.8\textwidth]{IntegrationTests.pdf}
       \caption{Workflow of the integration testing of PhEDEx, DBS 3, and DAS.}
 \label{fig:Integ-Phedex-DBS-DAS}
\end{figure}

\section{Conclusions}
The decision to base Data Management on independent core components, with a common user interface provided by the Data Aggregation Service, has brought CMS several advantages.
Each of the underlying services is based on the most appropriate technology and can be optimized independently for scalability, evolving without disrupting the overall system.

The components are interfaced to each other through a common CMSWEB web service framework, which simplifies integration and regression testing when rolling out new service versions.
Exposing the Data Management components through web service interfaces also allows to easily build external services that can integrate their information in a clean manner.

For example, the Victor \cite{victor} data cleaning service was developed in 2011 to identify data replicas that are no longer accessed, and can be deleted without disrupting user analysis.
Victor is interfaced to PhEDEx through the data service to discover the dataset replicas at each site and the overall space usage at the sites, and queries a dataset popularity service for the access frequency of file replicas. Combining these data, it can then provide lists of the least accessed dataset replicas to delete to free up space at full sites.
Looking ahead, another possible extension of the system would be an external dynamic data placement service that is able to request in PhEDEx new replicas of heavily accessed datasets, also querying the dataset popularity service.

In conclusion, CMS has developed a Data Management system that performed successfully during LHC Run 1, can be flexibly extended and is ready to manage the increased scale of data production during the second run of LHC.

\par
\section*{References}

\begin{thebibliography}{1}
\bibitem{PhEDEx}
  Egeland R, Wildish T and Metson S 2008 Data transfer infrastructure for CMS data taking,  {\it XII Advanced Computing and Analysis Techniques in Physics Research (Erice, Italy: Proceedings of Science)}

\bibitem{DBS}
Giffels M, Data Bookkeeping Service 3 - Providing event metadata in CMS, {\it submitted to CHEP 2013}

\bibitem{DAS}
Kuznetsov V, Evans D and Metson S, The CMS Data Aggregation System,
{\it doi:10.1016/j.procs.2010.04.172}

\bibitem{CMSDMS}
Giffels M, Guo Y, Kuznetsov V, Magini N and Wildish T, The CMS Data Management System {\it submitted to CHEP 2013}

%\bibitem
%The Perl Object Environment (POE)  {\it http://poe.perl.org/}

\bibitem{IPv6}
Campana S et.al. 2013 WLCG and IPv6 - the HEPiX IPv6 working group {\it submitted to CHEP 2013}
 
\end{thebibliography}


\end{document}

