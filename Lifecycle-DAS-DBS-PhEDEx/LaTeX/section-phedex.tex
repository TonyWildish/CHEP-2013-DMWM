\subsection{Scale tests of PhEDEx}
The LifeCycle agent was first written to perform scale tests of PhEDEx. We use a fake transfer layer consisting of a script that mimics FTS responses, using the input file-sizes and a timestamp in a cache to allow it to respond according to realistic expectations for throughput.

We use a topology that resembles the full CMS system (number of Tier-0/1/2 nodes, links between them), we configure data-flows that resemble the real activities of the experiment, and we run the full PhEDEx system on a number of machines (virtual or batch).

This lets us emulate realistic CMS conditions without needing network, disk or tape resources, so we can run at `data rates' far higher than expected in the near future. This has shown that PhEDEx will not hit any scaling problems in the next year or two if circumstances remain the same. If we did find a scaling problem, we will have time to address it before it happens in the production system.

\subsection{Testing new PhEDEx releases}

The LifeCycle agent is used to test PhEDEx releases. A private PhEDEx instance is set up in a testbed with a few nodes, and fake transfers between these nodes are driven using the LifeCycle agent to orchestrate the flow. In this way bug fixes and new features of new PhEDEx versions can be tested before releasing the version. 

For special cases, such as bugs that occur under unusual consitions, the
LifeCycle agent can be configured to reproduce these conditions, repeatedly
and systematically.


\subsection{Integration test of the web-based PhEDEx DataService}

One of the harder things to test is the website access by different users
with different access-rights. Rather than requiring the support of several
people with different credentials to systematically test new releases of the website, with all the
coordination and overhead that implies, we use the LifeCycle agent again.

We extend the private testbed to include a website, with self-signed
certificates allocated to specific roles. The LifeCycle agent uses the website
to perform a series of actions (create request, approve it, change its
priority), repeating this workflow for each certificate, and for the different
source and destination nodes. It checks that actions succeed where they should
and fail where appropriate. In this way, several combinations of actions
(inject, subscribe,...), destination nodes (T0, T1,...), and roles
(site-admin, data-manager,...) can be tested in an automated way by
configuring and running one single script.
