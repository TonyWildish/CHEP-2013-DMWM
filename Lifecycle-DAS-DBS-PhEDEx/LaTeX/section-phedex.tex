\subsection{Scale tests of PhEDEx}


\subsection{Testing new PhEDEx releases}

The LifeCycle agent is used to test PhEDEx releases. For this purpose a
private PhEDEx instance is set up in a testbed with a few nodes, and transfers
between these nodes are driven using the LifeCycle agent to orchestrate the
flow. In this procedure bug fixes and new features of new PhEDEx versions can
be tested before releasing the version. 

For special cases, such as bugs that occur under unusual consitions, the
LifeCycle agent can be configured to reproduce these conditions, repeatedly
and systematically.


\subsection{Integration test of the webbased PhEDEx DataService}


One of the harder things to test is the web-site access by different users
with different access-rights. Rather than requiring the support of several
people to systematically test the new release of the website, with all the
coordination and overhead that implies, we can use the LifeCycle agant again.

We extend the private testbed to include a website, with self-signed
certificates allocated to specific roles. The LifeCycle agent uses the website
to perform a series of actions (create request, approve it, change its
priority), repeating this workflow for each certificate, and for the different
source and destination nodes. It checks that actions succeed where they should
and fail where appropriate. In this way, several combinations of actions
(inject, subscribe,...), destination nodes (T0, T1,...), and roles
(site-admin, data-manager,...) can be tested in an automated way by
configuring and running one single script.


