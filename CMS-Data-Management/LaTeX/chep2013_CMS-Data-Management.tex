\documentclass[a4paper]{jpconf}
\usepackage{graphicx}
\usepackage{color}
\usepackage{array}
\usepackage{enumerate}

% for DAS section?
%\usepackage{amsmath}
%\usepackage{amstext}
%\usepackage{amssymb}
%\usepackage{graphicx}

\begin{document}
\title{The CMS Data Management System}

\author{M Giffels$^1$, Y Guo$^2$, V Kuznetsov$^3$,
        N Magini$^1$ and T Wildish$^4$}

\address{$^1$ CERN, CH-1211 Gen\`eve 23, Switzerland }
\address{$^2$ Fermi National Accelerator Laboratory, Batavia, Il, USA }
\address{$^3$ Cornell University, Ithaca, NY, USA }
\address{$^4$ Princeton University, Princeton, NJ, USA }

\ead{Nicolo.Magini@cern.ch}

\begin{abstract}
The data management elements in CMS are scalable, modular, and designed to work together. The main components are PhEDEx, the data transfer and location system; the Data Booking Service (DBS), a metadata catalog; and the Data Aggregation Service (DAS), designed to aggregate views and provide them to users and services. Tens of thousands of samples have been cataloged and petabytes of data have been moved since the run began. The modular system has allowed the optimal use of appropriate underlying technologies. In this contribution we will discuss the use of both Oracle and NoSQL databases to implement the data management elements as well as the individual architectures chosen. We will discuss how the data management system functioned during the first run, and what improvements are planned in preparation for 2015.
\end{abstract}

\section{Introduction}
The CMS \cite{CMS} experiment at the LHC has recently successfully concluded its first run. 
Between the fall of 2010 and the spring of 2013, over 10 billion raw data events were recorded, 
and 15 billion Monte Carlo events produced. The data were analysed using the distributed resources 
of CMS, managed through the Worldwide LHC Computing Grid (WLCG) \cite{WLCG} and through the CMS 
dataflow and workflow management tools.

The operational aspects of the CMS computing system are described separately \cite{OliOps}. In 
this paper we describe the performance of the CMS data management system during the first run of LHC. We describe 
what worked well and what didn't, and analyse the factors that contributed to the performance. 
Finally, we describe the improvements planned in the major subsystems for the second run of LHC, 
starting in 2015.


\section{Architecture}
\subsection{Components and data organisation}
The CMS data management system has several important tasks:
\begin{itemize}
  \item Data bookkeeping catalog - describes the data contents in physics terms
  \item Data location catalog - maintains knowledge of replicas of the data around CMS
  \item Data placement and transfer management
\end{itemize}

The core components which achieve these goals are:
\begin{itemize}
  \item PhEDEx \cite{PhEDEx}, the data-transfer management system
  \item DBS \cite{DBS}, the Data Bookeeping Service
  \item DAS \cite{DAS}, the Data Aggregation Service
\end{itemize}

There are a number of other, more specialised services, such as the RunRegistry, SiteDB, 
ConditionsDB and ReqMgr, but the core components are the ones the user mostly interacts with. All 
the components of the data management system are designed and constructed separately, interacting 
with each other and external users by means of web services.

CMS data is organised into datasets by physics content. Dataset size varies considerably, usually 
in the range of 0.1 - 100 TB. Within a dataset, files are organised into {\it blocks} of 10 - 1000 
files, purely as an aid to scalability. The data management tools can manipulate individual blocks 
of files, instead of entire datasets, which offers a finer granularity of control.

Files are typically about 2.5 GB in size. Small files from various processes are merged to 
maintain a minimum useful size to help scalability of storage and catalogs.

\subsection{Computing infrastructure}
The CMS computing infrastructure distinguishes between several tiers of computing centres, with a
single Tier-0, several Tier-1s, and a large number of Tier-2s.

The Tier-0 is responsible for custodial storage of a copy of the raw data, for calibration and prompt 
reconstruction and for distributing the raw and reconstructed data to the collaboration. The 
Tier-1s between them maintain a second custodial copy of the raw data, host Monte Carlo production and
re-reconstruction, and distribute data further to the Tier-2s.
The Tier-2 sites are where the physicists perform their analyses. Monte Carlo data 
produced at the Tier-2s is uploaded to the Tier-1s for custodial storage and distribution to other sites.

The original computing model \cite{CTDR} called for a hierarchical organisation of tiers, with each Tier-2 
attached to a single Tier-1 and each Tier-1 serving only a handful of Tier-2s. For our system (one 
Tier-0, 7 Tier-1s, and about 50 Tier-2s) this would mean less than 100 bidirectional data-transfer links. In 
practice, we now allow Tier-2s to connect to all Tier-1s and to all other Tier-2s, resulting in a 
fully-connected mesh of about 2000 data-transfer links.

\subsection{Data location and the Trivial File Catalog}
CMS does not use a central file catalog to store information about file location at each site. 
Instead, files are known by their {\it Logical File Name} (LFN), which maps to a different {\it 
Physical File Name} (PFN) at each site. We maintain the knowledge that a file exists at a site by 
storing the LFN-to-site mapping (in PhEDEx), but the mapping from LFN to PFN is the responsibility 
of each site.

To achieve this, each site maintains a {\it Trivial File Catalog} (TFC), which is essentially a 
regular-expression map from LFNs to PFNs and vice-versa. Anyone wishing to know the PFN for a file 
at a site simply processes the LFN through the regular expressions in that site's TFC to obtain 
their answer. This approach has several advantages:

\begin{itemize}
  \item Batch jobs running at the site do not need to contact central services to perform LFN to 
PFN conversions, they simply examine the local TFC,
  \item Sites can change their storage layout on the fly, without synchronising changes in each PFN
to a central service. As long as they maintain their TFC in sync with their storage, this is 
transparent to the rest of the system,
  \item Sites can perform complex matching for special situations. E.g. by matching the first 
letter of a GUID portion of an LFN, files can be load-balanced across several backend servers.
\end{itemize}


\section{PhEDEx}
The Physics Experiment Data Export (PhEDEx) \cite{PhEDEx} project was started in 2004 to manage global data transfers for CMS over the grid in a robust, reliable, and scalable way.

PhEDEx is based on a high-availability Oracle database cluster hosted at CERN (Transfer Management Data Base, or TMDB) acting as a ``blackboard'' for the system state (data replica location and current tasks). Users can request transfers of datasets or blocks of files through the interactive PhEDEx web site \cite{phedexweb}; a web data service \cite{phedexdatasvc} is also available for integration with other CMS data management components.

PhEDEx software daemon processes or ``agents'', based on the Perl Object Environment (POE) \cite{poe} framework, then  connect to the central database to retrieve their work queue, and write back to TMDB the result of their actions.
The TMDB has been carefully designed to minimize locking contention between the several agents and cache
coherency issues by using a row ``ownership'' model where only one specialized agent at a time is expected to act on a given set of rows.
Each of the agents performs a specific task, progressing the transfer state machine towards the desired final state: creating a new replica of the files at the destination.

Central agents running at CERN perform most of the intelligence of data routing and transfer task creation, expanding the blocks into their current file replicas, and calculating the path of least cost for each file from the available sources to the destination.
The download agents running at the sites fetch these tasks from TMDB and initiate the file transfers using specific plugins
for different grid middleware. To ensure reliability and robustness, each file transfer is independently verified, and intelligent backoff and retry policies are applied in case of failure, aiming for eventual completion of all transfer subscriptions.

Performance metrics are constantly recorded in TMDB, summarising snapshots of the TMDB state into dedicated status monitoring tables at regular intervals. Historical information is further aggregated from the status tables into time-series bins of data on transfer volume, transfer state counts, and number of failures. Since the website and data service only access these monitoring tables instead of the live tables to provide monitoring information, the performance of the user monitoring and of the transfer system are largely decoupled.

Additional data management actions that may be requested with PhEDEx and are executed by dedicated agents are data deletion and data consistency verification.

\subsection{PhEDEx performance}

%The scalability of PhEDEx for replica location is ensured my minimizing the amount of location metadata that needs to be tracked in TMDB. First of all, only the site location of the data needs to be recorded, since the details on the physical locations of the files are encoded in the sites'Trivial File Catalogs. In addition, PhEDEx needs to keep a long-term record only of the locations of block replicas: individual file replicas in a block are only needed while the block is in active transfer, and are collapsed again into the block after the block transfer is completed.

During the first run of LHC, CMS steadily transferred data with PhEDEx with peaks in global speed exceeding 5 GB/s, distributing more than 100 PB of replicas over all sites.


The scalability of PhEDEx as transfer management system was proven multiple times in the past with realistic simulations running in dedicated testbed instances, where the aggregated simulated transfer rates were pushed far beyond the requirements of CMS. The tool used to run the tests was the ``lifecycle agent'' \cite{lifecycle}, designed to exercise all components of the system for an extended period of time.
The lifecycle agent simulates the behavior of CMS data production components and users, regularly injecting new file replicas at various
nodes on a clone of the production infrastructure. It then subscribes data to other nodes for transfer, and requests deletion of some of the data.
The site download agents in the testbed are set up to execute fake transfers with a configurable bandwidth and failure probability, exercising error handling and retries.
During the latest stress test with the lifecycle agent in 2011, the system continued to work well running simulated transfers at rates at least ten times higher than the scale of production transfers.


\subsection{PhEDEx improvements}

PhEDEx was a mature product during data taking in LHC Run 1, and development was focused on adding support for more flexible workflows, and on providing more tools for the transfer operators.
PhEDEx 4.0 was released in early 2011 and added full support for transfers and deletions of individual blocks, allowing to subscribe only a part of a dataset. PhEDEx 4.1, deployed in 2012, included a new monitoring system for transfer latencies, which was described in \cite{phedexlatency}.
In this release cycle the core agent and namespace libraries were also gradually refactored \cite{phedexframework} and can now be used in other projects; the Namespace framework, in particular, was the base for the new storage space monitoring system \cite{storagemon}.
Current effort is dedicated to a new framework to handle user and operator requests, able for example to support requests to invalidate data \cite{phedexrequests}.


\section{DBS}
%!TEX root = chep2013_LifeCycle.tex
\subsection{Data Bookkeeping Service}
The Data Bookkeeping Service (DBS) \cite{DBS} provides a catalog of event metadata for Monte Carlo and recorded data of the CMS experiment. DBS contains records of what data exists, their process-oriented provenance information, including parentage relationships of files and datasets, their configurations of processing steps, as well as associations with run numbers and luminosity sections to find any particular subset of events within a dataset, on a large scale of about $200,000$ datasets and more than $40$ million files, which adds up in around $700$ GB of metadata. The DBS is an essential part of the CMS Data Management and Workload Management (DMWM) systems \cite{CMSDMS}, all kind of data-processing like Monte Carlo production, processing of recorded event data as well as physics analysis done by the users are heavily relying on the information stored in DBS.

\subsection{Scale testing of DBS~3}
The LifeCycle agent is perfectly suited for the scale testing of DBS~3, which is a crucial step to ensure a reliable system even in conditions of high load. DBS~3 is written in Python, therefore a Python framework was developed to simplify the interaction with the LifeCycle agent. The framework facilitates the handling of JSON payloads, error handling, building HTTP API calls from payloads and provides timing utilities. The collected data can be subsequently send back to the LifeCycle agent or exported to a SQLite database for further analysis. For the analysis of these data an additional package called LifeCycleAnalysis was developed. It contains a ROOT histogram manager automatically creating various timing and error distribution histograms for all the APIs used in the scale test. This framework is written generic and can be easily used by any python-based tool to interact with the LifeCycle agent. The actual scale tests were performed by using several LifeCycle agents and workflows submitted to a batch system, which allows to run an arbitrary number of workflows against the server being tested. Therefore, additional scripts are part of the framework to simplify the interaction with the LSF batch system at CERN.


\section{DAS}

The Data Aggregation Service (DAS) \cite{DAS} was designed to provide a uniform
access to distributed CMS data-services regardless of their security policies,
implementation details and data storage solution. DAS provides users with the ability to
query underlying data-services via the DAS Query Language (QL). It also aggregates
metadata information from various data-services into common records
suitable for end users. Here we discuss the details of DAS architecture and the
current status of the system.

DAS was built on top of a NoSQL document-oriented database (MongoDB
\cite{MongoDB}). It provides several benefits for the DAS use case: schema-less
storage, embedded query language and very fast read/write
operations.\footnote{Our benchmark showed that MongoDB can sustain 20k doc/s
for reading and 7k doc/s for writing operations, respectively.}
DAS has a modular design based on the MVC architecture \cite{MVC} and performs the
following workflow upon user query:

\begin{itemize}
\item fetch data from underlying data-services via existing APIs
\item store unprocessed data into the cache database
\item process data and perform its transformation to a common data representation
(we unify differences in naming conventions, units and data-formats, etc.)
\item aggregate data on a requested key, e.g. dataset, block, etc.
\item store results into the merge database
\item present results to end-users and provide a set of filters as well as
aggregation functions to perform basic operations for slicing and representing
the data in a form suitable for user tasks
\end{itemize}

This design allowed to develop DAS in a data-agnostic manner. For instance,
data can be fetched from underlying data-services regardless of their data format, e.g.
JSON, XML, CSV, etc. The data transformation was done via an external set of
mappings which was maintaned separately from DAS code development. It also gave
us a few benefits which were not foreseen from the design cycle. For example,
data aggregation on a common entity, e.g. data name, may show any existing discrepancy in
underlying services.

DAS uses an in-house Query Language which was based on entity relationships used by
physicists, see \cite{DAS}. It consists of the following structure:

\begin{verbatim}
<selection keys> <set of conditions> | <filters> | <aggregators>
\end{verbatim}

The selection keys were based on well known entities such as dataset, block, file,
run. The conditions were formed via key=value pairs. DAS-QL
provides a limited set of filters such as grep, unique,
as well as set of common aggregation functions such as
sum, min, max. The former support conditional operators and grouping, while
the latter can be extended via custom map-reduce functions to support more complex
use cases. Therefore to express a question {\it Find me all datasets at a
given site and show only those which have size greater than 50} someone will need to
write a DAS query in the following way:

\begin{verbatim}
dataset site=XYZ | grep dataset.name, dataset.size>50
\end{verbatim}

It turns out that such flexibility was not always clear to some users,
mostly those who were unfamiliar with the DAS-QL syntax. Therefore a further
attempt was made to build native support for keyword queries on top of DAS-QL,
see \cite{DASKWS}.

DAS operates as an intelligent cache in front of CMS data-services. It stores
results into two caches upon a provided query. The raw-cache is used to store results from
data-services {\it as is}, while the merge-cache stores aggregated
records. The lifetime of the records is based on information provided by
data providers via HTTP headers. The record maintenance is done in a lazy
fashion, i.e. upon a new user query expired records are wiped out from the cache,
while new ones come in. DAS server performs many operations in parallel, e.g.
it sends concurrent HTTP requests to underlying data-services, processes and
stores
data using multiple threads, and runs multiple monitoring and pre-fetching
daemons. The server runs on a single 8-core node with 24 GB of memory
required for efficient MongoDB operations\footnote{MongoDB relies on indexes
fitted in RAM to provide its superior speed}.

The discussed modular design, flexible QL and NoSQL storage allows DAS to
aggregate information from distributed data-services without imposing any
requirements on them. DAS is able to deal with different security models,
various APIs, data-formats and naming conventions. Right now it uses dozens of CMS
data-services. Data are aggregated into JSON records based on common entity
keys so that it is possible to see information from multiple
data-services in a single record, e.g. run information comes from DBS, Condition DB and
RunRegistry and is represented as a single JSON document listing information from
three data-services. Daily load on the DAS server is constanly growing and has about 10k
queries/day with $\sim O(10M)$ records going in and out of the DAS cache.



\section{Conclusions}
% placeholder...


\par
\section*{References}

\begin{thebibliography}{1}
\bibitem{CMS}
The CMS Collaboration 2008 The CMS experiment at the CERN LHC {\it JINST {\bf 3} S08004}

\bibitem{WLCG}
Knoblock J {\it et al.} 2005 LHC Computing Grid Technical Design Report {\it CERN-LHCC-2005-024}

\bibitem{OliOps}
 Gutsche O {\it et.al.} CMS Computing Operations during Run 1 {\it submitted to CHEP 2013}

\bibitem{PhEDEx}
  Egeland R, Wildish T and Metson S 2008 Data transfer infrastructure for CMS data taking,  {\it XII Advanced Computing and Analysis Techniques in Physics Research (Erice, Italy: Proceedings of Science)}

\bibitem{DBS}
Giffels M, DBS paper, {\it submitted to CHEP 2013}

\bibitem{DAS}
Kuznetsov V, Evans D and Metson S, The CMS Data Aggregation System,
{\it doi:10.1016/j.procs.2010.04.172}

\bibitem{cmscomptdr} Bonacorsi D 2007 The CMS computing model {\it Nucl. Phys.} B {\it (Proc. Suppl.)} {\bf 172} 53-56
{\bf is this a good reference or should we use the original CTDR???}

\bibitem{MongoDB}
http://www.mongodb.org/

\bibitem{MVC} Model-view-controller architecture, see
http://en.wikipedia.org/wiki/Model-view-controller

\bibitem{DASKWS} Zemleris V and Kuznetsov V,
Keyword Search over Data Service Integration for Accurate Results,
{\it submitted to CHEP 2013}

\end{thebibliography}


\end{document}

