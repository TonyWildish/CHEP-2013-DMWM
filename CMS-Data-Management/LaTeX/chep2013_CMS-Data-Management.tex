\documentclass[a4paper]{jpconf}
\usepackage{graphicx}
\usepackage{color}
\usepackage{array}
\usepackage{enumerate}

% for DAS section?
%\usepackage{amsmath}
%\usepackage{amstext}
%\usepackage{amssymb}
%\usepackage{graphicx}

\begin{document}
\title{The CMS Data Management System}

\author{M Giffels$^1$, Y Guo$^2$, V Kuznetsov$^3$,
        N Magini$^1$ and T Wildish$^4$}

\address{$^1$ CERN, CH-1211 Gen\`eve 23, Switzerland }
\address{$^2$ Fermi National Accelerator Laboratory, Batavia, Il, USA }
\address{$^3$ Cornell University, Ithaca, NY, USA }
\address{$^4$ Princeton University, Princeton, NJ, USA }

\ead{Nicolo.Magini@cern.ch}

\begin{abstract}
The data management elements in CMS are scalable, modular, and designed to work together. The main components are PhEDEx, the data transfer and location system; the Dataset Booking System (DBS), a metadata catalogue; and the Data Aggregation Service (DAS), designed to aggregate views and provide them to users and services. Tens of thousands of samples have been catalogued and petabytes of data have been moved since the run began. The modular system has allowed the optimal use of appropriate underlying technologies. In this contribution we will discuss the use of both Oracle and NoSQL databases to implement the data management elements as well as the individual architectures chosen. We will discuss how the data management system functioned during the first run, and what improvements are planned in preparation for 2015.
\end{abstract}

\section{Introduction}
PhEDEx \cite{PhEDEx}, DBS \cite{DBS} and DAS \cite{DAS} are the core components of the CMS Data 
Management system \cite{CMSDMS}. PhEDEx is responsible for moving data around the CMS computing 
sites, DBS is responsible for maintaining knowledge of the physics content of the data, and DAS is 
a portal that provides physicists with integrated access to PhEDEx, DBS, and other services.

The development model for these services is highly decoupled. Each is developed by a separate team 
of people, and each provides a web-based data service for access to information. Ensuring 
stability and consistency of these interfaces has, until recently, been poorly managed. We have 
relied on stability of the underlying services and coordination between developers to ensure that 
successive releases of the software are coherent and correct.

In this paper we describe part of the solution to this problem. The {\it LifeCycle agent}, written 
originally for scale-testing of PhEDEx, has been extended to provide cross-component integrated 
testing in a flexible and powerful manner. Components can be tested individually or together in 
different ways for verification, debugging, or scale and performance testing. The LifeCycle agent 
itself scales well, and is easily adapted to new requirements (e.g. testing of new APIs from the 
services). It can inject data into the component systems, retrieve it, manipulate it, and compare 
it with expectations to detect errors. It can also test expected error conditions for correct 
behaviour (e.g. access with an unauthorised certificate).


\section{Architecture}
\subsection{Components and data organisation}
The CMS data management system has several important tasks:
\begin{itemize}
  \item Data bookkeeping catalog - describes the data contents in physics terms
  \item Data location catalog - maintains knowledge of replicas of the data around CMS
  \item Data placement and transfer management
\end{itemize}

The core components which achieve these goals are:
\begin{itemize}
  \item PhEDEx \cite{PhEDEx}, the data-transfer management system
  \item DBS \cite{DBS}, the Data Bookeeping Service
  \item DAS \cite{DAS}, the Data Aggregation Service
\end{itemize}

There are a number of other, more specialised services, such as the RunRegistry, SiteDB, 
ConditionsDB and ReqMgr, but the core components are the ones the user mostly interacts with. All 
the components of the data management system are designed and constructed separately, interacting 
with each other and external users by means of web services.

CMS data is organised into datasets by physics content. Dataset size varies considerably, usually 
in the range of 0.1 - 100 TB. Within a dataset, files are organised into {\it blocks} of 10 - 1000 
files, purely as an aid to scalability. The data management tools can manipulate individual blocks 
of files, instead of entire datasets, which offers a finer granularity of control.

Files are typically about 2.5 GB in size. Small files from various processes are merged to 
maintain a minimum useful size to help scalability of storage and catalogs.

\subsection{Computing infrastructure}
The CMS computing infrastructure distinguishes between several tiers of computing centres, with a
single Tier-0, several Tier-1s, and a large number of Tier-2s.

The Tier-0 is responsible for custodial storage of a copy of the raw data, for calibration and prompt 
reconstruction and for distributing the raw and reconstructed data to the collaboration. The 
Tier-1s between them maintain a second custodial copy of the raw data, host Monte Carlo production and
re-reconstruction, and distribute data further to the Tier-2s.
The Tier-2 sites are where the physicists perform their analyses. Monte Carlo data 
produced at the Tier-2s is uploaded to the Tier-1s for custodial storage and distribution to other sites.

The original computing model \cite{CTDR} called for a hierarchical organisation of tiers, with each Tier-2 
attached to a single Tier-1 and each Tier-1 serving only a handful of Tier-2s. For our system (one 
Tier-0, 7 Tier-1s, and about 50 Tier-2s) this would mean less than 100 bidirectional data-transfer links. In 
practice, we now allow Tier-2s to connect to all Tier-1s and to all other Tier-2s, resulting in a 
fully-connected mesh of about 2000 data-transfer links.

\subsection{Data location and the Trivial File Catalog}
CMS does not use a central file catalog to store information about file location at each site. 
Instead, files are known by their {\it Logical File Name} (LFN), which maps to a different {\it 
Physical File Name} (PFN) at each site. We maintain the knowledge that a file exists at a site by 
storing the LFN-to-site mapping (in PhEDEx), but the mapping from LFN to PFN is the responsibility 
of each site.

To achieve this, each site maintains a {\it Trivial File Catalog} (TFC), which is essentially a 
regular-expression map from LFNs to PFNs and vice-versa. Anyone wishing to know the PFN for a file 
at a site simply processes the LFN through the regular expressions in that site's TFC to obtain 
their answer. This approach has several advantages:

\begin{itemize}
  \item Batch jobs running at the site do not need to contact central services to perform LFN to 
PFN conversions, they simply examine the local TFC,
  \item Sites can change their storage layout on the fly, without synchronising changes in each PFN
to a central service. As long as they maintain their TFC in sync with their storage, this is 
transparent to the rest of the system,
  \item Sites can perform complex matching for special situations. E.g. by matching the first 
letter of a GUID portion of an LFN, files can be load-balanced across several backend servers.
\end{itemize}


\section{PhEDEx}
\subsection{Scale tests of PhEDEx}


\subsection{Testing new PhEDEx releases}

The LifeCycle agent is used to test PhEDEx releases. For this purpose a
private PhEDEx instance is set up in a testbed with a few nodes, and transfers
between these nodes are driven using the LifeCycle agent to orchestrate the
flow. In this procedure bug fixes and new features of new PhEDEx versions can
be tested before releasing the version. 

For special cases, such as bugs that occur under unusual consitions, the
LifeCycle agent can be configured to reproduce these conditions, repeatedly
and systematically.


\subsection{Integration test of the webbased PhEDEx DataService}


One of the harder things to test is the web-site access by different users
with different access-rights. Rather than requiring the support of several
people to systematically test the new release of the website, with all the
coordination and overhead that implies, we can use the LifeCycle agant again.

We extend the private testbed to include a website, with self-signed
certificates allocated to specific roles. The LifeCycle agent uses the website
to perform a series of actions (create request, approve it, change its
priority), repeating this workflow for each certificate, and for the different
source and destination nodes. It checks that actions succeed where they should
and fail where appropriate. In this way, several combinations of actions
(inject, subscribe,...), destination nodes (T0, T1,...), and roles
(site-admin, data-manager,...) can be tested in an automated way by
configuring and running one single script.




\section{DBS}
%!TEX root = chep2013_CMS-Data-Management.tex
The Data Bookkeeping Service (DBS) \cite{DBS} provides a catalogue of event metadata for Monte Carlo and recorded data of the CMS experiment. DBS contains record of what data exist, their process-oriented provenance information, including parentage relationships of files and datasets, their configurations of processing steps, as well as associations with run numbers and luminosity sections to find any particular subset of events within a dataset, on a large scale of about $200,000$ datasets and more than $40$ million files, which adds up to around $700$ GB of metadata.

The current DBS, DBS~2 \cite{DBS2} was designed in 2006-2007, before the LHC started its operation. CMS did not have a standardised service architecture for the implementation, deployment and operation of that kind of web services. Thus DBS~2 was implemented using Java servlets in an Apache Tomcat container and XML RPC has been the first choice for the client-server communication. As  persistent storage system of the metadata, an Oracle database backend provided for CMS \cite{CMSDBs} is utilised. DBS~2 additionally supports a MySQL database backend, however it is currently not used in the production environment.

The DBS is a federated system with multiple instances for different scopes. Driven by the principle in CMS to separate official and user-created data, the Global-DBS contains metadata related to all official CMS data, real or simulated. Metadata related to user-created datasets can be stored in two physics analysis DBS instances. Besides those instances there are also a CAF instance, that records data from the prompt reconstruction stream used for detector calibrations and diagnostics, and a CMS Tier0 DBS instance, that records information for data from the detector as it is processed by the Tier0 facility.

Although DBS~2 sustained the load in LHC Run $1$, it has in some cases very ``thick'' client APIs, which  led to numerous problems with API versioning and scalability issues. In addition, the CMS data processing model has evolved a lot, in a way that could not be anticipated by the DBS~2 design back in $2006$, so many requests were made to store additional data in DBS~2, which were not entirely consistent with its original purpose. A project review in $2009$ led to the decision to re-design DBS, to better match the CMS data processing model and to better integrate with the DMWM projects.

The CMS DMWM project has meanwhile developed a standardised architecture based on the Representational State Transfer API (REST) \cite{REST} for its web services based on Python, CherryPy and SQLAlchemy. Thus DBS~3 has been re-designed and re-implemented in Python utilizing the CMS DMWM standards for RESTful web services. The client-server communication is stateless and REST also imposes the discipline of thin clients, which enhances the scalability of the service. The Java Script Object Notation (JSON) is used as data-format for the client-server communication being a lightweight replacement for XML RPC. The database schema of DBS~3 also has been revised, based on the experiences with DBS~2. The schema has been denormalised, since the DBS~2 schema was fully normalised, leading to excessive table joins and lock contention. In DBS~3 some tables were removed, as well as some relationships that were better modeled outside of the DBS schema. These changes sped up searches and also improved the insertion of data by removing foreign keys and lock contention. In addition, the DBS~3 particularly benefitted from the narrower and more precise project scope compared to DBS~2. The integration with other services (PhEDEx and DAS) in the CMS DMWM project made a narrower scope possible, without impairing the query features, so for example the data location is not anymore stored in DBS, since it is naturally available from PhEDEx. DBS~3 is currently deployed in parallel to DBS~2 for validation and integration with other DMWM projects. The final switch to DBS~3 is expected by the end of $2013$.



\section{DAS}

The Data Aggregation Service (DAS) \cite{DAS} was designed to provide a uniform
access to distributed CMS data-services regardless of their security policies,
implementation details and data storage solution. DAS provides users with the ability to
query underlying data-services via the DAS Query Language (QL). It also aggregates
metadata information from various data-services into common records
suitable for end users. Here we discuss the details of DAS architecture and the
current status of the system.

DAS was built on top of a NoSQL document-oriented database (MongoDB
\cite{MongoDB}). It provides several benefits for the DAS use case: schema-less
storage, embedded query language and very fast read/write
operations.\footnote{Our benchmark showed that MongoDB can sustain 20k doc/s
for reading and 7k doc/s for writing operations, respectively.}
DAS has a modular design based on the MVC architecture \cite{MVC} and performs the
following workflow upon user query:

\begin{itemize}
\item fetch data from underlying data-services via existing APIs
\item store unprocessed data into the cache database
\item process data and perform its transformation to a common data representation
(we unify differences in naming conventions, units and data-formats, etc.)
\item aggregate data on a requested key, e.g. dataset, block, etc.
\item store results into the merge database
\item present results to end-users and provide a set of filters as well as
aggregation functions to perform basic operations for slicing and representing
the data in a form suitable for user tasks
\end{itemize}

This design allowed to develop DAS in a data-agnostic manner. For instance,
data can be fetched from underlying data-services regardless of their data format, e.g.
JSON, XML, CSV, etc. The data transformation was done via an external set of
mappings which was maintaned separately from DAS code development. It also gave
us a few benefits which were not foreseen from the design cycle. For example,
data aggregation on a common entity, e.g. data name, may show any existing discrepancy in
underlying services.

DAS uses an in-house Query Language which was based on entity relationships used by
physicists, see \cite{DAS}. It consists of the following structure:

\begin{verbatim}
<selection keys> <set of conditions> | <filters> | <aggregators>
\end{verbatim}

The selection keys were based on well known entities such as dataset, block, file,
run. The conditions were formed via key=value pairs. DAS-QL
provides a limited set of filters such as grep, unique,
as well as set of common aggregation functions such as
sum, min, max. The former support conditional operators and grouping, while
the latter can be extended via custom map-reduce functions to support more complex
use cases. Therefore to express a question {\it Find me all datasets at a
given site and show only those which have size greater than 50} someone will need to
write a DAS query in the following way:

\begin{verbatim}
dataset site=XYZ | grep dataset.name, dataset.size>50
\end{verbatim}

It turns out that such flexibility was not always clear to some users,
mostly those who were unfamiliar with the DAS-QL syntax. Therefore a further
attempt was made to build native support for keyword queries on top of DAS-QL,
see \cite{DASKWS}.

DAS operates as an intelligent cache in front of CMS data-services. It stores
results into two caches upon a provided query. The raw-cache is used to store results from
data-services {\it as is}, while the merge-cache stores aggregated
records. The lifetime of the records is based on information provided by
data providers via HTTP headers. The record maintenance is done in a lazy
fashion, i.e. upon a new user query expired records are wiped out from the cache,
while new ones come in. DAS server performs many operations in parallel, e.g.
it sends concurrent HTTP requests to underlying data-services, processes and
stores
data using multiple threads, and runs multiple monitoring and pre-fetching
daemons. The server runs on a single 8-core node with 24 GB of memory
required for efficient MongoDB operations\footnote{MongoDB relies on indexes
fitted in RAM to provide its superior speed}.

The discussed modular design, flexible QL and NoSQL storage allows DAS to
aggregate information from distributed data-services without imposing any
requirements on them. DAS is able to deal with different security models,
various APIs, data-formats and naming conventions. Right now it uses dozens of CMS
data-services. Data are aggregated into JSON records based on common entity
keys so that it is possible to see information from multiple
data-services in a single record, e.g. run information comes from DBS, Condition DB and
RunRegistry and is represented as a single JSON document listing information from
three data-services. Daily load on the DAS server is constanly growing and has about 10k
queries/day with $\sim O(10M)$ records going in and out of the DAS cache.



\section{Conclusions}
The decision to base Data Management on independent core components, with a common user interface provided by the Data Aggregation Service, has brought CMS several advantages.
Each of the underlying services is based on the most appropriate technology and can be optimized independently for scalability, evolving without disrupting the overall system.

The components are interfaced to each other through a common CMSWEB web service framework, which simplifies integration and regression testing when rolling out new service versions.
Exposing the Data Management components through web service interfaces also allows to easily build external services that can integrate their information in a clean manner.

For example, the Victor \cite{victor} data cleaning service was developed in 2011 to identify data replicas that are no longer accessed, and can be deleted without disrupting user analysis.
Victor is interfaced to PhEDEx through the data service to discover the dataset replicas at each site and the overall space usage at the sites, and queries a dataset popularity service for the access frequency of file replicas. Combining these data, it can then provide lists of the least accessed dataset replicas to delete to free up space at full sites.
Looking ahead, another possible extension of the system would be an external dynamic data placement service that is able to request in PhEDEx new replicas of heavily accessed datasets, also querying the dataset popularity service.

In conclusion, CMS has developed a Data Management system that performed successfully during LHC Run 1, can be flexibly extended and is ready to manage the increased scale of data production during the second run of LHC.

\par
\section*{References}

\begin{thebibliography}{1}
\bibitem{PhEDEx}
  Egeland R, Wildish T and Metson S 2008 Data transfer infrastructure for CMS data taking,  {\it XII Advanced Computing and Analysis Techniques in Physics Research (Erice, Italy: Proceedings of Science)}

\bibitem{DBS}
Giffels M, Data Bookkeeping Service 3 - Providing event metadata in CMS, {\it submitted to CHEP 2013}

\bibitem{DAS}
Kuznetsov V, Evans D and Metson S, The CMS Data Aggregation System,
{\it doi:10.1016/j.procs.2010.04.172}

\bibitem{CMSDMS}
Giffels M, Guo Y, Kuznetsov V, Magini N and Wildish T, The CMS Data Management System {\it submitted to CHEP 2013}

%\bibitem
%The Perl Object Environment (POE)  {\it http://poe.perl.org/}

\bibitem{IPv6}
Campana S et.al. 2013 WLCG and IPv6 - the HEPiX IPv6 working group {\it submitted to CHEP 2013}
 
\end{thebibliography}


\end{document}

