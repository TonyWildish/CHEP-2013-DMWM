\subsection{Components and data organisation}
The CMS data management system has several important tasks:
\begin{itemize}
  \item Data bookkeeping catalog - describes the data contents in physics terms
  \item Data location catalog - maintains knowledge of replicas of the data around CMS
  \item Data placement and transfer management
\end{itemize}

The core components which achieve these goals are:
\begin{itemize}
  \item PhEDEx \cite{PhEDEx}, the data-transfer management system
  \item DBS \cite{DBS}, the Dataset Bookeeping Service
  \item DAS \cite{DAS}, the Data Aggregation Service
\end{itemize}

There are a number of other, more specialised services, such as the RunRegistry, SiteDB, 
ConditionsDB and ReqMgr, but the core components are the ones the user mostly interacts with. All 
the components of the data management system are designed and constructed separately, interacting 
with each other and external users by means of web-services.

CMS data is organised into datasets by physics content. Dataset size varies considerably, usually 
in the range of 0.1 - 100 TB. Within a dataset, files are organised into {\it blocks} of 10 - 1000 
files, purely as an aid to scalability. The data management tools can manipulate individual blocks 
of files, instead of entire datasets, which offers a finer granularity of control.

Files are typically about 2.5 GB in size. Small files from various processes are merged to 
maintain a minimum useful size to help scalability of storage and catalogues.

\subsection{Computing infrastructure}
The computing infrastructure has evolved since it was first described in the CMS Computing TDR 
\cite{CTDR}. We still have the distinction between tiers, with a Tier-0, several Tier-1s, and a 
large number of Tier-2s.

The Tier-0 is responsible for custodial storage of a copy of the raw data, for prompt 
reconstruction and for distributing the raw and reconstructed data to the collaboration. The 
Tier-1s between them maintain a second custodial copy of the raw data, and distribute data further 
to the Tier-2s. The Tier-2 sites are where the physicists perform their analyses. Monte carlo data 
is produced at the Tier-2s and uploaded to the Tier-1s for distribution to other sites.

The original computing model called for a hierarchical organisation of tiers, with each Tier-2 
attached to a single Tier-1 and each Tier-1 serving only a handful of Tier-2s. For our system (one 
Tier-0, 7 Tier-1s, and about 50 Tier-2s) this would mean less than 100 data-transfer links. In 
practice, we now allow Tier-2s to connect to all Tier-1s and to all other Tier-2s, resulting in a 
fully-connected mesh of aboutt 2000 data-transfer links.

\subsection{Data location and the Trivial File Catalog}
CMS does not use a central file catalog to store information about file location at each site. 
Instead, files are known by their {\it Logical File Name} (LFN), which maps to a different {\it 
Physical File Name} (PFN) at each site. We maintain the knowledge that a file exists at a site by 
storing the LFN-to-site mapping (in PhEDEx), but the mapping from LFN to PFN is the responsibility 
of each site.

To achieve this, each site maintains a {\it Trivial File Catalog} (TFC), which is essentially a 
regular-expression map from LFNs to PFNs and vice-versa. Anyone wishing to know the PFN for a file 
at a site simply processes the LFN through the regular expressions in that sites TFC to obtain 
their answer. This approach has several advantages:

\begin{itemize}
  \item Batch jobs running at the site do not need to contact central services to perform LFN to 
PFN conversions, they simply examine the local TFC
  \item Sites can change their storage layout on the fly, without synchronising updates to a 
central service. As long as they maintain their TFC in sync with their storage, this is 
transparent to the rest of the system
  \item Sites can perform complex matching for special situations. E.g. by matching the first 
letter of a GUID portion of an LFN files can be load-balanced across several backend servers
\end{itemize}
